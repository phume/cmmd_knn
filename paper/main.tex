\documentclass[conference]{IEEEtran}

\usepackage{graphicx}
\usepackage{amsmath}
\usepackage{amssymb}
\usepackage{booktabs}
\usepackage{hyperref}
\usepackage{algorithm}
\usepackage{algorithmic}

\title{PNKDIF: Peer-Normalized Kernel Deep Isolation Forest for Training-Free Contextual Anomaly Detection}

\author{
\IEEEauthorblockN{Author Name}
\IEEEauthorblockA{Affiliation\\
Email}
}

\begin{document}

\maketitle

\begin{abstract}
Contextual anomaly detection aims to identify samples whose behavioral features deviate from expectations given their context. Existing methods face a trade-off: training-based approaches like Conditional VAEs can model complex context-behavior relationships but require optimization and risk overfitting, while training-free methods like QCAD and ROCOD avoid these issues but are limited to linear decision boundaries. We propose Peer-Normalized Kernel Deep Isolation Forest (PNKDIF), a training-free method that achieves non-linear decision boundaries. PNKDIF first identifies contextually similar peers via K-nearest neighbors with RBF kernel weighting, then normalizes behavioral features against peer statistics. The normalized features are projected through multiple frozen randomly-initialized MLPs, creating diverse non-linear representations. Isolation Forest scores on each representation are aggregated to produce the final anomaly score. Our approach combines the expressiveness of deep learning with the simplicity and robustness of training-free methods. We demonstrate competitive performance on benchmark datasets while maintaining $O(N \log N)$ computational complexity and requiring no hyperparameter tuning for learning.
\end{abstract}

\section{Introduction}

Anomaly detection is a fundamental task in machine learning with applications spanning fraud detection, network intrusion detection, medical diagnosis, and quality control. Traditional approaches treat all samples uniformly, comparing each observation against global statistics or decision boundaries learned from the entire dataset. However, this global perspective fails when the definition of ``normal'' varies systematically across different contexts.

Consider fraud detection in financial transactions: a \$50,000 wire transfer may be perfectly normal for a corporate account but highly suspicious for a college student. Similarly, a heart rate of 180 bpm is alarming at rest but expected during intense exercise. In both cases, the behavioral features (transaction amount, heart rate) must be interpreted relative to the context features (account type, activity level) to accurately assess anomalousness.

This observation motivates \emph{contextual anomaly detection} (CAD), where the goal is to identify samples whose behavioral features deviate from what is expected given their context. Formally, rather than modeling the marginal distribution $P(Y)$, we seek to model the conditional distribution $P(Y|C)$ and flag samples where $y$ is unlikely given $c$.

Existing CAD methods face a fundamental trade-off between expressiveness and simplicity:
\begin{itemize}
    \item \textbf{Training-based methods} such as Conditional Variational Autoencoders (CVAE) \cite{sohn2015learning} and Context-aware Wasserstein Autoencoders (CWAE) can learn complex non-linear relationships between context and behavior, but require careful architecture design, hyperparameter tuning, and may overfit or learn to ignore rare anomaly patterns.
    \item \textbf{Training-free methods} such as ROCOD \cite{liang2023rocod} and QCAD \cite{liang2021conditional} avoid these pitfalls but are limited to linear decision boundaries in the behavioral feature space, potentially missing anomalies with complex multivariate structure.
\end{itemize}

In this paper, we propose \textbf{Peer-Normalized Kernel Deep Isolation Forest (PNKDIF)}, a method that achieves the best of both worlds: non-linear decision boundaries without training. Our approach builds on three key ideas:

\begin{enumerate}
    \item \textbf{Kernel-weighted peer normalization}: We identify each sample's contextually similar peers via K-nearest neighbors and compute RBF kernel-weighted statistics. Z-score normalization against these peer statistics removes context-dependent location and scale effects.

    \item \textbf{Frozen random projections}: Inspired by random feature methods \cite{rahimi2007random} and Deep Isolation Forest \cite{xu2023deep}, we project normalized features through multiple frozen randomly-initialized MLPs. These create diverse non-linear representations without any optimization.

    \item \textbf{Ensemble Isolation Forest scoring}: We apply Isolation Forest to each projected representation and aggregate scores. The axis-aligned cuts of IF in the transformed space correspond to non-linear boundaries in the original space.
\end{enumerate}

\noindent\textbf{Contributions.} Our main contributions are:
\begin{itemize}
    \item We propose PNKDIF, the first training-free contextual anomaly detection method with non-linear decision boundaries.
    \item We introduce kernel-weighted peer normalization, which provides soft peer weighting via RBF kernels rather than hard K-NN cutoffs.
    \item We demonstrate that frozen random MLP projections, combined with Isolation Forest, can capture complex anomaly patterns without any learned parameters.
    \item We provide theoretical analysis of computational complexity and empirical evaluation on benchmark datasets.
\end{itemize}

The remainder of this paper is organized as follows. Section~\ref{sec:related} reviews related work in anomaly detection. Section~\ref{sec:method} presents the PNKDIF algorithm in detail. Section~\ref{sec:experiments} describes our experimental setup and results. Section~\ref{sec:discussion} discusses design choices and limitations. Section~\ref{sec:conclusion} concludes.

\section{Related Work}
\label{sec:related}

\subsection{Traditional Anomaly Detection}

Classical anomaly detection methods operate on the full feature space without distinguishing context from behavior. \textbf{Isolation Forest (IF)} \cite{liu2008isolation} isolates anomalies by recursively partitioning data with random axis-aligned splits; points requiring fewer splits to isolate are considered more anomalous. IF is efficient ($O(N \log N)$) and requires no density estimation, but its axis-aligned cuts limit expressiveness in the original feature space.

\textbf{Local Outlier Factor (LOF)} \cite{breunig2000lof} compares each point's local density to its neighbors, flagging points in sparser regions. While LOF is inherently local, it does not distinguish between context and behavioral features---all dimensions contribute equally to neighbor selection and density estimation.

\textbf{One-Class SVM} \cite{scholkopf2001estimating} learns a decision boundary enclosing normal data in kernel space. Though capable of non-linear boundaries, it requires careful kernel selection and is computationally expensive for large datasets.

\subsection{Deep Learning for Anomaly Detection}

\textbf{Deep Isolation Forest (DIF)} \cite{xu2023deep} extends IF by first projecting data through randomly-initialized neural networks, then applying IF to the transformed representations. The random projections create non-linear isolation boundaries without training. Our method adapts this idea for the contextual setting, applying random projections after peer normalization.

\textbf{Autoencoders} detect anomalies via reconstruction error, with variants including Variational Autoencoders (VAE) \cite{kingma2014auto} and adversarial approaches \cite{zenati2018adversarially}. These methods learn global representations and do not naturally accommodate context-dependent behavior.

\subsection{Contextual Anomaly Detection}

\textbf{QCAD} \cite{liang2021conditional} (Quantile-based Conditional Anomaly Detection) estimates conditional quantiles of behavioral features given context via K-NN. Points outside expected quantile ranges are flagged as anomalies. QCAD is training-free but limited to detecting univariate deviations in each behavioral dimension independently.

\textbf{ROCOD} \cite{liang2023rocod} (Robust Conditional Outlier Detection) extends peer-based methods with robust statistics and density-weighted combinations. Like QCAD, it uses hard K-NN peer selection and linear (location-scale) normalization.

\textbf{Conditional VAE/CVAE} \cite{sohn2015learning} conditions the latent space on context features, learning to reconstruct behavior given context. Anomalies have high reconstruction error. While expressive, CVAEs require substantial training data and careful architecture design.

\textbf{Context-aware Wasserstein Autoencoder (CWAE)} uses optimal transport objectives for conditional generation, providing sharper density estimates than VAE-based methods. However, training remains a bottleneck.

\textbf{Normalcy Score (NS)} methods compute z-scores relative to predicted behavior, where predictions come from regression or other supervised models trained on normal data. These approaches assume access to clean training data and labeled contexts.

\textbf{ConQuest}~\cite{calikus2025context} approaches contextual anomaly detection from a different angle: automatic context discovery. Rather than assuming known context/behavior splits, ConQuest uses multi-objective optimization to find informative context subsets. It also introduces Shared Nearest Neighbors (SNN) distance for peer weighting, which is more robust in high-dimensional spaces than Euclidean distance.

\subsection{Random Feature Methods}

\textbf{Random Fourier Features} \cite{rahimi2007random} approximate shift-invariant kernels via random projections, enabling kernel methods to scale to large datasets. This theoretical foundation---that random projections can approximate complex functions---motivates our use of frozen MLPs.

\textbf{Extreme Learning Machines} \cite{huang2006extreme} use single-layer networks with random hidden weights, training only the output layer. Our approach goes further by eliminating all training, using random projections purely for feature transformation before IF scoring.

\subsection{Positioning of PNKIF}

Table~\ref{tab:comparison} positions PNKIF relative to existing methods. Our method uniquely combines: (1) context-awareness via peer normalization, and (2) training-free operation via Isolation Forest.

\begin{table}[h]
\centering
\caption{Comparison of anomaly detection methods}
\label{tab:comparison}
\begin{tabular}{@{}lcc@{}}
\toprule
Method & Context-Aware & Training-Free \\
\midrule
IF \cite{liu2008isolation} & \texttimes & \checkmark \\
DIF \cite{xu2023deep} & \texttimes & \checkmark \\
QCAD \cite{liang2021conditional} & \checkmark & \checkmark \\
ROCOD \cite{liang2023rocod} & \checkmark & \checkmark \\
ConQuest \cite{calikus2025context} & \checkmark & \checkmark \\
CVAE \cite{sohn2015learning} & \checkmark & \texttimes \\
CWAE & \checkmark & \texttimes \\
\textbf{PNKIF (Ours)} & \checkmark & \checkmark \\
\bottomrule
\end{tabular}
\end{table}

Unlike QCAD and ROCOD which use uniform K-NN weighting, PNKIF employs RBF kernel weighting for smoother peer statistics. PNKIF-SNN adopts ConQuest's Shared Nearest Neighbors approach for better performance in high-dimensional context spaces. Unlike deep methods (CVAE, CWAE), PNKIF requires no training and avoids overfitting risks.

\section{Methodology}

% Write your methodology section here

\section{Experiments}
\label{sec:experiments}

\subsection{Experimental Setup}

\subsubsection{Datasets}
We evaluate PNKDIF on synthetic and real-world datasets:

\begin{itemize}
    \item \textbf{Synthetic}: Controlled datasets with known ground truth, including linear shift, scale variation, multimodal clusters, and nonlinear manifold anomalies.
    \item \textbf{UCI Adult}: Census income data with demographics as context and work-related features as behavior.
    \item \textbf{UCI Bank}: Marketing campaign data with customer profile as context and campaign statistics as behavior.
    \item \textbf{Cardio (ODDS)}: Cardiology dataset with patient demographics as context and clinical measurements as behavior.
\end{itemize}

\begin{table}[h]
\centering
\caption{Dataset statistics}
\label{tab:datasets}
\begin{tabular}{@{}lrrrr@{}}
\toprule
Dataset & $N$ & $d_c$ & $d_y$ & Anomaly \% \\
\midrule
Syn-Linear & 10,000 & 2 & 5 & 5\% \\
Syn-Scale & 10,000 & 2 & 5 & 5\% \\
Syn-Multimodal & 10,000 & 2 & 5 & 5\% \\
Syn-Nonlinear & 10,000 & 2 & 5 & 5\% \\
Adult (shift) & 30,162 & 5 & 4 & 5\% \\
Bank (shift) & 30,488 & 7 & 9 & 5\% \\
Cardio & 1,831 & 5 & 16 & 9.6\% \\
\bottomrule
\end{tabular}
\end{table}

\subsubsection{Baselines}
We compare against the following methods:
\begin{itemize}
    \item \textbf{Isolation Forest (IF)}: Standard IF on behavioral features only.
    \item \textbf{IF-concat}: IF on concatenated context and behavioral features.
    \item \textbf{Deep Isolation Forest (DIF)}: IF with random MLP projections on behavior.
    \item \textbf{DIF-concat}: DIF on concatenated features.
    \item \textbf{QCAD}: Quantile-based conditional anomaly detection~\cite{liang2021conditional}.
    \item \textbf{ROCOD}: Robust conditional outlier detection with peer-based statistics~\cite{liang2023rocod}.
    \item \textbf{LOF}: Local Outlier Factor on concatenated features~\cite{breunig2000lof}.
\end{itemize}

\subsubsection{Evaluation Metrics}
We report:
\begin{itemize}
    \item \textbf{AUROC}: Area under the ROC curve, measuring ranking quality.
    \item \textbf{AUPRC}: Area under the Precision-Recall curve, appropriate for imbalanced data.
    \item \textbf{P@k}: Precision at top-$k$ ranked samples ($k=100$).
\end{itemize}

\subsubsection{Implementation Details}
For PNKDIF, we use $K=100$ neighbors, $M=6$ projections, $d_h=128$ hidden dimensions, $T=100$ trees, and $\psi=256$ subsample size. The kernel bandwidth $\gamma$ is set via the median heuristic. All experiments are run with 5 random seeds and we report mean $\pm$ standard deviation.

\subsection{Results}

\subsubsection{Synthetic Data Performance}

Table~\ref{tab:auroc_synthetic} shows results on synthetic datasets designed to test specific aspects of contextual anomaly detection.

\begin{table}[h]
\centering
\caption{AUROC on synthetic datasets. Best in bold, second-best underlined.}
\label{tab:auroc_synthetic}
\begin{tabular}{@{}lcccc@{}}
\toprule
Method & Linear & Scale & Multimodal & Nonlinear \\
\midrule
IF & 0.660 & 0.967 & 0.501 & 0.820 \\
IF-concat & 0.984 & 0.993 & \underline{0.980} & 0.918 \\
DIF & 0.658 & 0.968 & 0.499 & 0.814 \\
DIF-concat & \underline{0.992} & 0.989 & 0.970 & 0.903 \\
LOF & 0.578 & 0.619 & 0.559 & \underline{0.999} \\
QCAD & 0.990 & 0.990 & 0.528 & 0.962 \\
ROCOD & \textbf{1.000} & \underline{0.999} & 0.676 & \textbf{1.000} \\
\midrule
\textbf{PNKDIF} & \textbf{1.000} & \textbf{0.999} & \textbf{0.867} & \textbf{1.000} \\
\bottomrule
\end{tabular}
\end{table}

\textbf{Key findings}: (1) On linear shift anomalies, PNKDIF and ROCOD achieve near-perfect detection while standard IF fails (0.66 AUROC). (2) PNKDIF significantly outperforms all methods on multimodal data (+22\% over IF-concat), demonstrating the benefit of peer-aware normalization. (3) On nonlinear manifold anomalies, PNKDIF achieves perfect detection.

Figure~\ref{fig:synthetic} visualizes the comparison.

\begin{figure}[h]
\centering
\includegraphics[width=\columnwidth]{figures/synthetic_comparison.pdf}
\caption{AUROC comparison on synthetic datasets. Red border indicates best method.}
\label{fig:synthetic}
\end{figure}

\subsubsection{Real Data Performance}

Table~\ref{tab:auroc_real} presents results on real-world datasets with synthetic anomaly injection (shift-type).

\begin{table}[h]
\centering
\caption{AUROC on real datasets with shift injection. Best in bold.}
\label{tab:auroc_real}
\begin{tabular}{@{}lccc@{}}
\toprule
Method & Adult & Bank & Cardio \\
\midrule
IF & 0.993 & \textbf{0.999} & \textbf{0.949} \\
DIF & 0.986 & 0.996 & 0.943 \\
LOF & 0.705 & 0.886 & 0.547 \\
QCAD & 0.993 & \textbf{0.999} & 0.709 \\
ROCOD & 0.942 & 0.935 & 0.768 \\
\midrule
PNKDIF & 0.988 & 0.996 & 0.744 \\
PNKDIF-noMLP & \textbf{0.996} & \textbf{0.999} & 0.780 \\
\bottomrule
\end{tabular}
\end{table}

\textbf{Observations}: (1) On shift-injected anomalies, most methods achieve high AUROC (>0.93), with PNKDIF-noMLP matching or exceeding IF. (2) On Cardio, IF outperforms PNKDIF significantly (0.949 vs 0.780), suggesting the anomalies in this dataset are global outliers rather than contextual. (3) The simpler PNKDIF-noMLP variant often performs best, indicating that random MLP projections may not help when behavioral dimensionality is already low.

\begin{figure}[h]
\centering
\includegraphics[width=\columnwidth]{figures/real_data_comparison.pdf}
\caption{AUROC on real datasets with shift-type anomaly injection.}
\label{fig:real}
\end{figure}

\subsubsection{Ablation Studies}

We evaluate the contribution of each PNKDIF component on the Syn-Nonlinear dataset (Table~\ref{tab:ablation}).

\begin{table}[h]
\centering
\caption{Ablation study on Syn-Nonlinear dataset}
\label{tab:ablation}
\begin{tabular}{@{}lc@{}}
\toprule
Variant & AUROC \\
\midrule
PNKDIF (full) & \textbf{1.0000} \\
w/o kernel weighting (uniform) & 1.0000 \\
w/o MLP projection & 0.9998 \\
w/o peer normalization (DIF) & 0.8145 \\
\bottomrule
\end{tabular}
\end{table}

\textbf{Insights}: Peer normalization is the critical component, providing +18.5\% AUROC over DIF. The kernel weighting and MLP projections provide marginal improvements on this dataset. On real data with low-dimensional behavior, the MLP projection may even hurt performance.

\begin{figure}[h]
\centering
\includegraphics[width=0.7\columnwidth]{figures/ablation.pdf}
\caption{Ablation study showing the contribution of each component.}
\label{fig:ablation}
\end{figure}

\subsubsection{Scalability}

Figure~\ref{fig:scalability} shows wall-clock runtime as dataset size increases from $N=1,000$ to $N=50,000$.

\begin{figure}[h]
\centering
\includegraphics[width=0.8\columnwidth]{figures/scalability.pdf}
\caption{Runtime vs. dataset size. PNKDIF scales as $O(N \log N)$.}
\label{fig:scalability}
\end{figure}

PNKDIF maintains competitive runtime, processing 50K samples in ~4 seconds. The overhead versus DIF is ~25\% due to the K-NN step, which is amortized by the efficiency of ball-tree data structures.

\subsubsection{Hyperparameter Sensitivity}

We study sensitivity to the key hyperparameter $K$ (number of neighbors):

\begin{table}[h]
\centering
\caption{Sensitivity to $K$ (AUROC on Syn-Linear)}
\label{tab:k_sensitivity}
\begin{tabular}{@{}lccccc@{}}
\toprule
$K$ & 10 & 25 & 50 & 100 & 200 \\
\midrule
AUROC & 0.985 & 0.996 & 0.999 & \textbf{1.000} & 1.000 \\
\bottomrule
\end{tabular}
\end{table}

PNKDIF is robust across a wide range of $K$ values (25-200). Too small $K$ (<25) leads to noisy peer statistics, while very large $K$ (>500) loses locality. The number of projections $M$ and hidden dimension $d_h$ have minimal impact; even $M=1$ achieves strong performance.

\begin{figure}[h]
\centering
\includegraphics[width=\columnwidth]{figures/hyperparameter_sensitivity.pdf}
\caption{Hyperparameter sensitivity analysis.}
\label{fig:hyperparameter}
\end{figure}

\section{Discussion}
\label{sec:discussion}

\subsection{Design Rationale}

\subsubsection{Why K-NN for Peer Selection?}
K-nearest neighbors provides a non-parametric approach to identifying contextually similar samples. Unlike radius-based methods, K-NN guarantees a fixed peer group size regardless of local density variations. This is important for stable statistics computation---radius-based neighbors can produce empty or extremely large peer groups depending on context space density. K-NN also yields interpretable peer groups: ``the K most similar entities'' is a domain-meaningful concept in applications like fraud detection or anomaly monitoring.

\subsubsection{Why RBF Kernel Weighting Instead of Uniform?}
While K-NN identifies the peer set, uniform weighting treats the nearest and K-th nearest neighbor equally. RBF kernel weighting introduces soft boundaries: closer peers contribute more to the mean and variance estimates. This has two benefits: (1) it avoids discontinuities at the K-th neighbor cutoff, producing smoother anomaly score surfaces as points move in context space; (2) it down-weights peers that are contextually similar but not identical, reducing the influence of marginally relevant reference points.

\subsubsection{Why Z-Score Normalization?}
Z-score normalization encodes the assumption that context affects behavior through location (shift) and scale effects. This assumption holds in many practical settings: larger companies have larger transactions, athletes have higher baseline heart rates, etc. The z-score has a direct interpretation---``2.5 standard deviations above the peer mean''---which aids explainability. Alternative approaches like quantile normalization are more robust to non-Gaussian distributions but lose magnitude information that may be relevant for anomaly detection.

\subsubsection{Why Frozen Random MLPs?}
Training neural networks for anomaly detection faces a fundamental challenge: without labeled anomalies, what objective should be optimized? Autoencoders learn to reconstruct the training distribution, potentially learning to reconstruct (and thus miss) rare anomalies. Our frozen random projections sidestep this by making no assumptions about anomaly patterns. The theoretical justification comes from random feature methods \cite{rahimi2007random}: random projections can approximate kernel evaluations, providing non-linear decision boundaries without optimization. Each random initialization emphasizes different feature interactions, and the ensemble averages out pathological projections.

\subsubsection{Why Isolation Forest as Final Scorer?}
Isolation Forest is well-suited to high-dimensional transformed spaces where density estimation fails. Its $O(\log \psi)$ query time makes scoring efficient, and its path-length semantics are interpretable: anomalies are ``easier to isolate.'' In the MLP-transformed space, IF's axis-aligned cuts become curved boundaries in the original z-score space, capturing non-linear anomaly patterns.

\subsection{Limitations}

\subsubsection{Curse of Dimensionality in Context Space}
When context features are high-dimensional, K-NN becomes unreliable as distances concentrate and neighborhoods become sparse. This can lead to unstable peer statistics. Potential mitigations include dimensionality reduction on context features (e.g., PCA, autoencoders) or careful feature selection based on domain knowledge.

\subsubsection{Simple Context-Behavior Relationship}
Our z-score normalization assumes context affects behavior through shift and scale. This cannot capture complex conditional relationships where context changes the shape of the behavioral distribution (e.g., bimodal distributions in some contexts but unimodal in others). Deep learning approaches like CVAE can learn such complex mappings but at the cost of training requirements and potential overfitting.

\subsubsection{Pre-defined Distance Metric}
K-NN requires a meaningful distance metric in context space. For continuous features, Euclidean distance is standard, but mixed feature types (categorical + continuous) require careful encoding. The method does not learn an optimal metric, unlike metric learning approaches.

\subsubsection{Homogeneous Peer Edge Case}
When all K peers have identical behavioral values, the peer standard deviation is zero. We handle this with an $\epsilon$ floor, which results in very large z-scores for any deviation. This behavior is arguably correct---deviation from a perfectly homogeneous peer group is suspicious---but may produce extreme scores in edge cases.

\subsection{Hyperparameter Sensitivity}

The method introduces several hyperparameters:
\begin{itemize}
    \item $K$ (neighbors): Too small leads to noisy statistics; too large loses locality and context specificity. We recommend $K \in [50, 200]$ for typical dataset sizes.
    \item $\gamma$ (bandwidth): Too small means only the nearest neighbor matters; too large approaches uniform weighting. The median heuristic provides a reasonable default.
    \item $M$ (projections): Diminishing returns beyond $M \approx 6\text{--}10$. More projections increase robustness but also computation.
    \item $d_h$ (hidden dimension): Larger dimensions are more expressive but may overfit to noise in the random projection. We recommend $d_h \in [64, 256]$.
\end{itemize}

Empirically, PNKDIF is less sensitive to hyperparameters than training-based methods, as there is no learning rate, batch size, or early stopping to tune.

\subsection{Interpretability}

PNKDIF offers partial interpretability:
\begin{itemize}
    \item \textbf{Peer-level}: For any flagged sample, we can retrieve its peer group and show the peer mean/std, explaining what ``normal'' behavior looks like for similar contexts.
    \item \textbf{Z-score level}: The normalized features show which behavioral dimensions deviate most from peer expectations.
    \item \textbf{Score-level}: The final score indicates relative anomalousness within the dataset.
\end{itemize}

However, the random MLP projections and IF path lengths are not directly interpretable. Understanding \emph{why} IF flagged a point in the transformed space requires examining the learned (random) feature interactions, which is non-trivial.

\section{Conclusion}

% Write your conclusion here


\bibliographystyle{IEEEtran}
\bibliography{references}

\end{document}
